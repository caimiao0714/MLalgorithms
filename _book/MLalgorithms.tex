\documentclass[fontset=fandol,zihao=false,scheme=chinese,heading=true,UTF8]{ctexbook}
\usepackage{lmodern}
\usepackage{amssymb,amsmath}
\PassOptionsToPackage{dvipsnames}{xcolor}
\RequirePackage{xcolor} % [dvipsnames]
%\usepackage{xcolor}

\usepackage{ifxetex,ifluatex}
\usepackage{fixltx2e} % provides \textsubscript
\ifnum 0\ifxetex 1\fi\ifluatex 1\fi=0 % if pdftex
  \usepackage[T1]{fontenc}
  \usepackage[utf8]{inputenc}
\else % if luatex or xelatex
  \ifxetex
    \usepackage{xltxtra,xunicode}
  \else
    \usepackage{fontspec}
  \fi
  \defaultfontfeatures{Ligatures=TeX,Scale=MatchLowercase}
\fi
% use upquote if available, for straight quotes in verbatim environments
\IfFileExists{upquote.sty}{\usepackage{upquote}}{}
% use microtype if available
\IfFileExists{microtype.sty}{%
\usepackage{microtype}
\UseMicrotypeSet[protrusion]{basicmath} % disable protrusion for tt fonts
}{}
\usepackage[b5paper,paperwidth=17.6cm,paperheight=25cm,tmargin=3.2cm,bmargin=3.2cm,lmargin=2.5cm,rmargin=2.5cm]{geometry}
\usepackage[unicode=true]{hyperref}
\PassOptionsToPackage{usenames,dvipsnames}{color} % color is loaded by hyperref
\hypersetup{
            pdftitle={Machine learning Algorithms},
            pdfauthor={蔡苗},
            colorlinks=true,
            linkcolor=Maroon,
            citecolor=Blue,
            urlcolor=Blue,
            breaklinks=true}
\urlstyle{same}  % don't use monospace font for urls
\usepackage{natbib}
\bibliographystyle{plainnat}
\usepackage{longtable,booktabs}
% Fix footnotes in tables (requires footnote package)
\IfFileExists{footnote.sty}{\usepackage{footnote}\makesavenoteenv{long table}}{}
\IfFileExists{parskip.sty}{%
\usepackage{parskip}
}{% else
\setlength{\parindent}{0pt}
\setlength{\parskip}{6pt plus 2pt minus 1pt}
}
\setlength{\emergencystretch}{3em}  % prevent overfull lines
\providecommand{\tightlist}{%
  \setlength{\itemsep}{0pt}\setlength{\parskip}{0pt}}
\setcounter{secnumdepth}{5}
% Redefines (sub)paragraphs to behave more like sections
\ifx\paragraph\undefined\else
\let\oldparagraph\paragraph
\renewcommand{\paragraph}[1]{\oldparagraph{#1}\mbox{}}
\fi
\ifx\subparagraph\undefined\else
\let\oldsubparagraph\subparagraph
\renewcommand{\subparagraph}[1]{\oldsubparagraph{#1}\mbox{}}
\fi

% set default figure placement to htbp
\makeatletter
\def\fps@figure{htbp}
\makeatother

\usepackage{booktabs}
\usepackage{longtable}

\usepackage{framed,color}
\definecolor{shadecolor}{RGB}{248,248,248}
\usepackage{style/lshort-zn-cnMiao}

\renewcommand{\textfraction}{0.05}
\renewcommand{\topfraction}{0.8}
\renewcommand{\bottomfraction}{0.8}
\renewcommand{\floatpagefraction}{0.75}

\newenvironment{dedication}
{
   \cleardoublepage
   \thispagestyle{empty}
   \vspace*{\stretch{1}}
   \hfill\begin{minipage}[t]{0.66\textwidth}
   \raggedright
}
{
   \end{minipage}
   \vspace*{\stretch{3}}
   \clearpage
}

\let\oldhref\href
\renewcommand{\href}[2]{#2\footnote{\url{#1}}}

\makeatletter
\newenvironment{kframe}{%
\medskip{}
\setlength{\fboxsep}{.8em}
 \def\at@end@of@kframe{}%
 \ifinner\ifhmode%
  \def\at@end@of@kframe{\end{minipage}}%
  \begin{minipage}{\columnwidth}%
 \fi\fi%
 \def\FrameCommand##1{\hskip\@totalleftmargin \hskip-\fboxsep
 \colorbox{shadecolor}{##1}\hskip-\fboxsep
     % There is no \\@totalrightmargin, so:
     \hskip-\linewidth \hskip-\@totalleftmargin \hskip\columnwidth}%
 \MakeFramed {\advance\hsize-\width
   \@totalleftmargin\z@ \linewidth\hsize
   \@setminipage}}%
 {\par\unskip\endMakeFramed%
 \at@end@of@kframe}
\makeatother

%\renewenvironment{Shaded}{\begin{kframe}}{\end{kframe}}

\usepackage{makeidx}
\makeindex

\urlstyle{tt}

\usepackage{amsthm}
\makeatletter
\def\thm@space@setup{%
  \thm@preskip=8pt plus 2pt minus 4pt
  \thm@postskip=\thm@preskip
}
\makeatother

\frontmatter


\title{{\fontsize{26}{30}\selectfont \textbf{Machine learning Algorithms}}}
\author{蔡苗\footnote{Department of Epidemiology and Biostatistics, College for Public Health and Social Justice, Saint Louis University. Email: \url{miao.cai@slu.edu}}}
\date{2019-04-18}

\begin{document}
\maketitle

\begin{dedication}
感谢我的家人的支持。
\end{dedication}

\section*{Acknowledgement}

I want to thank my mentor.

{
\setcounter{tocdepth}{2}
\tableofcontents
}
\listoftables
\listoffigures



\hypertarget{preface}{%
\chapter{Preface}\label{preface}}

This book works as a notebook to summarize the algorithms used in Bayesian inference and machine learning.

\mainmatter

\hypertarget{introduction}{%
\chapter{Introduction}\label{introduction}}

\hypertarget{optimization}{%
\chapter{Optimization}\label{optimization}}

\hypertarget{discrete-optimization}{%
\section{Discrete optimization}\label{discrete-optimization}}

The \textbf{objective function} allows us to measure how ``good'' any given solution to the problem is.
We seek to maximize or minimize the objective function.

\textbf{Derivative/gradient} based methods keep going ``uphill'' until they are at the top of the h

\hypertarget{heuristic-and-metaheuristic-methods}{%
\subsection{Heuristic and metaheuristic methods}\label{heuristic-and-metaheuristic-methods}}

\begin{quote}
``a \textbf{heuristic} is a technique designed for solving a problem more quickly when classic methods are too slow, or for finding an approximate solution when classic methods fail to find any exact solution.''
\end{quote}

\begin{quote}
Wikipedia
\end{quote}

Heuristic methods do not guarantee to find the global optimal solution (best solution)!
Instead, they seek to find \textbf{a best available solution, given the resource spent looking for it}.
A \textbf{heuristic method} is \textbf{geared towards a specific problem}.

\begin{quote}
a \textbf{metaheuristic} is a higher-level procedure or heuristic designed to find, generate, or select a heuristic (partial search algorithm) that may provide a sufficiently good solution to an optimization problem, especially with incomplete or imperfect information or limited computation capacity. Metaheuristics sample a set of solutions which is too large to be completely sampled. Metaheuristics may make few assumptions about the optimization problem being solved, and so they may be usable for a variety of problems
\end{quote}

\begin{quote}
-- Wikipedia
\end{quote}

A metaheuristic method is like a heuristic, but generalizable to a broad class of problems.

\begin{enumerate}
\def\labelenumi{\arabic{enumi}.}
\tightlist
\item
  Genetic Algorithms (Holland -- 1975)
\end{enumerate}

\begin{itemize}
\tightlist
\item
  Natural selection / genetics based. Popular method.
\end{itemize}

\begin{enumerate}
\def\labelenumi{\arabic{enumi}.}
\setcounter{enumi}{1}
\tightlist
\item
  Simulated Annealing (Kirpatrick -- 1983)
\end{enumerate}

\begin{itemize}
\tightlist
\item
  Metallurgy annealing, find lowest energy level!
\end{itemize}

\begin{enumerate}
\def\labelenumi{\arabic{enumi}.}
\setcounter{enumi}{2}
\tightlist
\item
  Particle Swarm Optimization (Eberhart Kennedy - 1995)
\end{enumerate}

\begin{itemize}
\tightlist
\item
  Based on insect behavior, swarming towards optimal location (food). Less common in discrete spaces. originally proposed for continuous spaces.
\end{itemize}

\begin{enumerate}
\def\labelenumi{\arabic{enumi}.}
\setcounter{enumi}{3}
\tightlist
\item
  Tabu Search (Al-Sultan -- 1999)
\end{enumerate}

\begin{itemize}
\tightlist
\item
  Search for best neighborhood solution, then find new neighborhood. Prior neighborhoods are forbidden (tabu)
\end{itemize}

General meta-heuristics traits

\begin{itemize}
\tightlist
\item
  Evaluate many potential optimal solutions.
\item
  Evaluate the fitness of each solution based on a cost (objective) function.
\item
  Use some concept of stochastic (random) movement to generate new solutions from the parameter space.
\item
  Use some set of rules to determine where to move next in the parameter space.
\item
  Declare convergence once some set of criteria has been met. Perhaps no improvement for X iterations.
\end{itemize}

\hypertarget{genetic-algorithm-and-simulated-annealing-as-examples}{%
\subsection{Genetic algorithm and simulated Annealing as examples}\label{genetic-algorithm-and-simulated-annealing-as-examples}}

\textbf{Genetic algorithm}: need to explore large portions of the parameter space at random. Concept of ``neighbor'' is vague.

\href{https://toddwschneider.com/posts/traveling-salesman-with-simulated-annealing-r-and-shiny/}{A nice shiny app}

An GA example: Since a new treatment for Hep C has become available, where is the optimal place to locate limited new Hep C resources, considering where our patients live?

The problem become intractable with large number of locations and resources: How many combinations of patients and clinics can I calculate the full feature space for to find a maximum?

\begin{itemize}
\tightlist
\item
  Exact Solution is NP-Hard
\item
  Calculations = \(n^{\sqrt{k}}\)
\item
  I conveniently stopped my analysis at 6 sites with \textasciitilde{}5k patients, requiring 1,149,712,053 distance calculations (I have a big server)
\item
  The ``k-center'' problem
\end{itemize}

\textbf{Simulated Annealing}:

\begin{itemize}
\tightlist
\item
  The concept of a `neighbor' is strong.
\item
  Can be sensitive to parameter choice, or algorithm gets stuck in global minima!
\item
  Generally, you should try both to see what works best. Hard to guess up front.
\end{itemize}

\hypertarget{constrains}{%
\subsection{Constrains}\label{constrains}}

Hard constraints

\begin{itemize}
\tightlist
\item
  If this constraint is violated, we have invalid solution.
\item
  Labor laws, number of nurses available, etc
\end{itemize}

Soft Constraints

\begin{itemize}
\tightlist
\item
  These are nice to meet if possible (included in cost function somehow), but if they are not met the solution is still valid.
\item
  Nurse prefers to only work X night shifts per month.
\item
  Leave requests.
\end{itemize}

\hypertarget{introduction-1}{%
\chapter{Introduction}\label{introduction-1}}

\bibliography{bib/bib.bib}


\backmatter
\printindex

\end{document}
